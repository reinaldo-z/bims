\newcommand{\seccion}{SECUNDARIA INCORPORADA A LA SEG }
\newcommand{\descripcion}{Ecuaciones lineales}
\newcommand{\grado}{Primero de secundaria}
\newcommand{\ciclo}{Ciclo escolar: 2015--2016}
\newcommand{\papel}{letterpaper} %letterpaper, legalpaper ...
\newcommand{\fecha}{19 de febrero de 2016}

\author{M. en C. Reinaldo Zapata}

\documentclass[11pt]{article}
\usepackage[\papel]{geometry}

\title{\vspace{-1cm}\flushleft \seccion \\ \descripcion \\  \grado \\ \ciclo}

\newcommand\BackgroundLogo{
\put(160,285){
\parbox[b][\paperheight]{\paperwidth}{%
\vfill
\centering
\includegraphics[width=5cm,height=2.5cm,keepaspectratio]{/Users/reinaldo/Documents/clases/jassa/logo}%
\vfill
}}}

% \hyphenation{con-ti-nua-ci\'on}

\usepackage{enumitem}
\usepackage[T1]{fontenc} %fuentes
\usepackage{lmodern} %fuente mejorada
\usepackage[spanish]{babel}
\decimalpoint
\usepackage{fullpage}
\usepackage{graphicx}
\usepackage{eso-pic}
\usepackage{multirow}
\usepackage{subfigure}
\usepackage{tikz}
\usepackage{hyperref} 
\usepackage{color}
\usepackage{multicol}
\usepackage{tikz}
\usetikzlibrary{shapes.geometric}



\usepackage[leqno,fleqn]{amsmath}
\makeatletter
  \def\tagform@#1{\maketag@@@{#1\@@italiccorr}}
\makeatother
\renewcommand{\theequation}{\fbox{\textbf{\arabic{equation}}}}


\begin{document}
\AddToShipoutPicture*{\BackgroundLogo}
\ClearShipoutPicture
\date{\fecha}
\maketitle


% \begin{minipage}[t]{0.8\linewidth}
Nombre del alumno:\,\line(1,0){244}\,.\hspace*{.2cm} \hfill Aciertos:\,\line(1,0){35}\,. \\
\indent Primero de secundaria, grupo:\,\line(1,0){35}\,. No. de lista:\,\line(1,0){35}\,. \hfill 20 \quad \ 
% \end{minipage}
% \begin{minipage}{0.8\linewidth}
% \end{minipage}

\vspace{5mm}

Siguiendo el procedimiento visto en clase resuelve las ecuaciones que se
presentan a continuac\'on.

\begin{multicols}{2}

\begin{equation}
5x = 10
\end{equation}

\vspace{1cm}

\begin{equation}
15y = 15
\end{equation}

\vspace{1cm}

\begin{equation}
4z = 12
\end{equation}

\vspace{1cm}

\begin{equation}
7w = 14
\end{equation}

\vspace{1cm}

\begin{equation}
2a = 1
\end{equation}

\vspace{1cm}

\begin{equation}
15x = 90
\end{equation}

\vspace{1cm}

\begin{equation}
25y = 200
\end{equation}

\vspace{1cm}

\begin{equation}
5x = 10
\end{equation}

\vspace{1cm}


\end{multicols}

\newpage

\begin{multicols}{2}

\begin{equation}
3y = 7.5
\end{equation}

\vspace{1cm}

\begin{equation}
4x = 1
\end{equation}

\vspace{1cm}

\begin{equation}
2x + 10 = 30
\end{equation}

\vspace{1cm}

\begin{equation}
5y - 15 = 5
\end{equation}

\vspace{1cm}

\begin{equation}
8z + 10 = 50
\end{equation}

\vspace{1cm}

\begin{equation}
20w - 10 = 90
\end{equation}

\vspace{1cm}

\begin{equation}
10 + 2x = 20
\end{equation}

\vspace{1cm}

\begin{equation}
30 - 5y = 25
\end{equation}

\vspace{1cm}

\begin{equation}
20 - 3z = 5
\end{equation}

\vspace{1cm}

\begin{equation}
25a - 50 = 0
\end{equation}

\vspace{1cm}

\begin{equation}
3x + 5 + 2x = 25
\end{equation}

\vspace{1cm}

\begin{equation}
7y - 5 + 3y = 35
\end{equation}

\vspace{1cm}



\end{multicols}

    
\end{document}










