\newcommand{\seccion}{SECUNDARIA INCORPORADA A LA SEG }
\newcommand{\descripcion}{Introducci\'on al \'algebra}
\newcommand{\grado}{Primero de secundaria}
\newcommand{\ciclo}{Ciclo escolar: 2015--2016}
\newcommand{\papel}{letterpaper} %letterpaper, legalpaper ...
\newcommand{\fecha}{29 de enero de 2016}

\author{M. en C. Reinaldo Zapata}

\documentclass[11pt]{article}
\usepackage[\papel]{geometry}

\title{\vspace{-1cm}\flushleft \seccion \\ \descripcion \\  \grado \\ \ciclo}

\newcommand\BackgroundLogo{
\put(160,285){
\parbox[b][\paperheight]{\paperwidth}{%
\vfill
\centering
\includegraphics[width=5cm,height=2.5cm,keepaspectratio]{/Users/reinaldo/Documents/clases/jassa/logo}%
\vfill
}}}

% \hyphenation{con-ti-nua-ci\'on}

\usepackage{enumitem}
\usepackage[T1]{fontenc} %fuentes
\usepackage{lmodern} %fuente mejorada
\usepackage[spanish]{babel}
\decimalpoint
\usepackage{fullpage}
\usepackage{graphicx}
\usepackage{eso-pic}
\usepackage{multirow}
\usepackage{subfigure}
\usepackage{tikz}
\usepackage{hyperref} 
\usepackage{color}
\usepackage{multicol}
\usepackage{tikz}
\usetikzlibrary{shapes.geometric}



\usepackage[leqno,fleqn]{amsmath}
\makeatletter
  \def\tagform@#1{\maketag@@@{#1\@@italiccorr}}
\makeatother
\renewcommand{\theequation}{\fbox{\textbf{\arabic{equation}}}}


\begin{document}
\AddToShipoutPicture*{\BackgroundLogo}
\ClearShipoutPicture
\date{\fecha}
\maketitle


% \begin{minipage}[t]{0.8\linewidth}
Nombre del alumno:\,\line(1,0){244}\,.\hspace*{.2cm} \hfill Aciertos:\,\line(1,0){35}\,. \\
\indent Primero de secundaria, grupo:\,\line(1,0){35}\,. No. de lista:\,\line(1,0){35}\,. \hfill 20 \quad \ 
% \end{minipage}
% \begin{minipage}{0.8\linewidth}
% \end{minipage}

\vspace{5mm}

Completa la tabla que se muestra a continuaci\'on separando las expresiones
algebraicas en sus respectivos componentes.

\begin{center}
{\large
\begin{tabular}{|c|c|c|c|}
\hline
Expresi\'on & Coeficiente & Literal(es) & Exponente(s)  \\ \hline 
$3x^2$ &&&\\ \hline
$b^2c^4x^6$ &&&\\ \hline
$17ax^2y^3$ &&&\\ \hline
$d^2we^3x$ &&&\\ \hline
\end{tabular}
}
\end{center}

Siguiendo el procedimiento visto en clase haz la reducci\'on de t\'erminos
semejantes en las expresiones que se muestran a continuaci\'on.


\begin{multicols}{2}
    
\begin{equation}
3m + 20p + 5m - 6p = 
\end{equation}
\begin{equation}
6p + 8m - 4p - 2m = 
\end{equation}
\begin{equation}
5p + m -p +2m + 6p =
\end{equation}
\begin{equation}
8p + m -5p + 12m + 2p + 2m =
\end{equation}
\begin{equation}
3x + 4y - x = 
\end{equation}
\begin{equation}
20x + 17y - 8x - y -3x =
\end{equation}

\end{multicols}
\newpage
\begin{multicols}{2}

\begin{equation}
4mp + 5m -3mp +8p - 2m =
\end{equation}
\begin{equation}
2n + 77m =
\end{equation}
\begin{equation}
7xy + 45x -6xy - 2x + 8y = 
\end{equation}
\begin{equation}
7abc + 5xyz - abc + 6xyz = 
\end{equation}
\begin{equation}
-ax + by + 2ax - 3by = 
\end{equation}
\begin{equation}
5xy + 7ab - 3xy -6ab - xy = 
\end{equation}
\begin{equation}
30x + 8xy - 4xy - 12x + -2b =
\end{equation}
\begin{equation}
-x + 7x -y + 8y = 
\end{equation}


\end{multicols}

    
\end{document}










