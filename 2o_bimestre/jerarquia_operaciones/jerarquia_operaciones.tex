\newcommand{\seccion}{SECUNDARIA INCORPORADA A LA SEG }
\newcommand{\descripcion}{Jerarqu\'iaa de Operaciones }
\newcommand{\grado}{Primero de secundaria}
\newcommand{\ciclo}{Ciclo escolar: 2015--2016}
\newcommand{\papel}{legalpaper} %letterpaper, legalpaper ...
\newcommand{\fecha}{30 de noviembre de 2015}

\author{M. en C. Reinaldo Zapata}

\documentclass[11pt]{article}
\usepackage[\papel]{geometry}

\title{\flushleft \seccion \\ \descripcion \\  \grado \\ \ciclo}

\newcommand\BackgroundLogo{
\put(162,365){
\parbox[b][\paperheight]{\paperwidth}{%
\vfill
\centering
\includegraphics[width=5cm,height=2.5cm,keepaspectratio]{/Users/reinaldo/Documents/clases/jassa/logo}%
\vfill
}}}

% \hyphenation{con-ti-nua-ci\'on}

\usepackage{enumitem}
\usepackage[T1]{fontenc} %fuentes
\usepackage{lmodern} %fuente mejorada
\usepackage[spanish]{babel}
\decimalpoint
\usepackage{fullpage}
\usepackage{multicol}
\usepackage{graphicx}
\usepackage{eso-pic}
\usepackage{multirow}
\usepackage{subfigure}
\usepackage{tikz}
\usepackage{hyperref} 
\usepackage{color}
\usepackage{multicol}
\usepackage[leqno,fleqn]{amsmath}
\makeatletter
  \def\tagform@#1{\maketag@@@{#1\@@italiccorr}}
\makeatother
\renewcommand{\theequation}{\fbox{\textbf{\arabic{equation}}}}


\begin{document}
\AddToShipoutPicture*{\BackgroundLogo}
\ClearShipoutPicture
\date{\today}
\maketitle

A continuaci\'on se presentan una serie de ejercicios con n\'umeros decimales y
fraccionarios. Resuelve cada una de las operaciones respetando la jerarqu\'ia de
las operaciones. Si tienes dudas revisa los apuntes de clase o usa
\href{https://www.youtube.com/watch?v=KDDcZCvgx5k}{\color{blue}{este
v\'inclulo}}.

\begin{multicols}{2}
    
\section{Fracciones}

\begin{align}
\left(  \frac{1}{3} + \frac{2}{30} \right) \div \frac{1}{6} =&&  2\frac{2}{5} \\
\left( 8 + \frac{3}{4} \div 4\frac{1}{5} \right) = && 2\frac{1}{12} \\
\left( 4 - \frac{1}{3} \div \frac{11}{6} \right) = && 2 \\
\frac{3}{5} \div \left( \frac{2}{3} + \frac{5}{6} \right)= && \frac{2}{5} \\
\frac{9}{10} \div \left( 2\frac{1}{3} - 1\frac{1}{4} \right) = && \frac{54}{65} \\
\left( \frac{1}{2} + \frac{3}{4} - \frac{1}{8} \right) \div 1\frac{3}{5} = && \frac{45}{64} \\
\left( 2\frac{1}{3} + 3\frac{1}{4} - 3\frac{1}{8} \right) \div \frac{1}{12} = && 29\frac{1}{2} \\
\left( 5 \div \frac{1}{5} \right) \div \left( 2 \div \frac{1}{3} \right) = && 4\frac{1}{6} \\
\left( 2 \times \frac{6}{5} \right) \div \left( 2 + \frac{3}{8} \right) = && 1\frac{1}{95} \\
\left( \frac{7}{30} + \frac{7}{90} + \frac{1}{3} \right) \div \frac{1}{9} = && 5\frac{4}{5}
\end{align}


\section{Decimales}

\begin{align}
(2.5 + 8)(1.4) = && 14.7 \\
(4.7 - 3) \div (8 - 4.5) = && 0.485 \\
7 + 8 \times 4.6 = && 43.8 \\
(2 \div 5) \times (4 - 1.023) = && 1.1908 \\
(7 + 4.545 - 3) \times (1 - 0.85) \div 2 = && 0.64
\end{align}

\end{multicols}

\section{N\'umeros al cuadrado}
Desarrolla los siguientes n\'umeros al cuadrado haciendo cada operaci\'on. 
\begin{align}
10^{2}  &= \qquad 100 \\ 
15^{2}  &= \qquad 225 \\
25^{2}  &= \qquad 625 \\
4.23625^{2} &= \qquad 17.8929\\
\pi^{2} &= \qquad 9.86965056
\end{align}




\end{document}

\left(  \right)




