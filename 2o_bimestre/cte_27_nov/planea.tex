\newcommand{\seccion}{SECUNDARIA INCORPORADA A LA SEG }
\newcommand{\descripcion}{An\'alisis de resultados de Planea }
\newcommand{\grado}{Primero de secundaria}
\newcommand{\ciclo}{Ciclo escolar: 2015--2016}
\newcommand{\papel}{legalpaper} %letterpaper, legalpaper ...
\newcommand{\fecha}{30 de noviembre de 2015}

\author{M. en C. Reinaldo Zapata}

\documentclass[11pt]{article}
\usepackage[\papel]{geometry}

\title{\flushleft \seccion \\ \descripcion \\  \grado \\ \ciclo}

\newcommand\BackgroundLogo{
\put(162,365){
\parbox[b][\paperheight]{\paperwidth}{%
\vfill
\centering
\includegraphics[width=5cm,height=2.5cm,keepaspectratio]{/Users/reinaldo/Documents/clases/jassa/logo}%
\vfill
}}}

% \hyphenation{con-ti-nua-ci\'on}

\usepackage{enumitem}
\usepackage[T1]{fontenc} %fuentes
\usepackage{lmodern} %fuente mejorada
\usepackage[spanish]{babel}
\decimalpoint
\usepackage{fullpage}
\usepackage{multicol}
\usepackage{graphicx}
\usepackage{eso-pic}
\usepackage{multirow}
\usepackage{subfigure}
\usepackage{tikz}

\begin{document}
\AddToShipoutPicture*{\BackgroundLogo}
\ClearShipoutPicture
\date{\fecha}
\maketitle
% \thispagestyle{empty}


Respecto a los resultados de Planea se pueden observar las siguientes
tendencias: cuando nos comparamos con otras escuelas similares a la nuestra y a
las escuelas del pa\'is:

\vspace{1cm}

Nivel I y II: el porcentaje de alumnos que otras instituciones tienen en nivel I
es mayor que el nuestro; tenemos m\'as alumnos en nivel II que otras
instituciones. Nivel III y IV: el porcentaje de alumnos que otras instituciones
tienen en nivel IV es superior al nuestro.

\vspace{1cm}

Esto lo podemos interpretar de la manera siguiente: nuestros alumnos est\'an
colocados principalmente en el nivel II no siendo \'este el nivel deseado al
finalizar la secundaria. 

\vspace{1cm}

En mi posici\'on actual como docente en nuestra instituci\'on me corresponde elevar 
el nivel b\'asico en los conocimientos de los alumnos para que as\'i ellos puedan 
desarrollar competencias de nivel superior al pasar a segundo o tercero de 
secundaria.

\vspace{1cm}

Acorde a lo evaluado en Planea los niveles que me corresponden son los
siguientes: 

\vspace{1cm}

\hfill%
\begin{itemize}
\item \textbf{Nivel I:} Resuelven problemas que implican comparar o realizar c\'alculos con
n\'umeros naturales.

A este primer nivel le corresponden los siguientes aprendizajes b\'asicos del perfil de egreso de primero de secundaria:

\qquad N\'umeros primos y criterios de divisibilidad.

\qquad Suma, resta, multiplicaci\'on y divisi\'on con fracciones y decimales.

\qquad Orden de la resoluci\'on de operaciones de acuerdo a su jerarqu\'ia.

\qquad Problemas que implican operaciones b\'asicas con fracciones y decimales.

\qquad Ubicaci\'on de cantidades enteras, decimales y fraccionarias en rectas num\'ericas.

\item \textbf{Nivel II:} Resuelven problemas con n\'umeros decimales y ecuaciones lineales
sencillas.

A este segundo nivel le corresponden los siguientes aprendizajes b\'asicos del perfil de egreso de primero de secundaria:

\qquad C\'alculo de per\'imetros y \'areas de figuras planas regulares e irregulares.

\qquad Resoluci\'on de problemas que implican per\'imetros y \'areas de figuras planas.

\qquad Concepto de variable.

\qquad Identificaci\'on de t\'erminos semejantes en una expres\'on algebraica 

\qquad Resoluci\'on y planteamiento de ecuaciones lineales de primer grado con una inc\'ognita.

\newpage

\item \textbf{Nivel III:} (s\'olo una parte): Resuelven problemas con n\'umeros fraccionarios o
naturales con signo.

A este tercer nivel le corresponden los siguientes aprendizajes b\'asicos del perfil de egreso de primero de secundaria:

\qquad Proporcionalidad directa.

\qquad Regla de tres y sus usos.

\qquad Porcentajes.

\qquad Suma y resta con fracciones usando n\'umeros con signo.

\qquad Suma y resta con decimales usando n\'umeros con signo.



\end{itemize}

\end{document}
