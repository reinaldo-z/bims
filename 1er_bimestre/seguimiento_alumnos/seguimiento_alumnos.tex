\documentclass{article}
\usepackage{fullpage}
\usepackage{graphicx}
\usepackage{eso-pic}

\title{{\sc Seguimiento de Alumnos} \\ Primer bimestre}
\author{M. en C. Reinaldo Arturo Zapata Pe\~na}
\date{}

\newcommand\BackgroundLogo{
\put(162,335){
\parbox[b][\paperheight]{\paperwidth}{%
\vfill
\centering
\includegraphics[width=5cm,height=2.5cm,keepaspectratio]{/Users/reinaldo/Documents/clases/jassa/logo}%
\vfill
}}}


\begin{document}
\AddToShipoutPicture*{\BackgroundLogo}
\ClearShipoutPicture

\maketitle

\section*{\sc Primeoro de secundaria, grupo A}

{\Large Calificaciones menores a 6}
\vspace{3mm}

\textbf{Cordero Lopez Eugenio:} La calificaci\'on del examen bimestral muestra mejoras en su desempe\~no. Si contin\'ua poniendo de su parte en clase ser\'a suficiente para tener una calificaci\'on aprobatoria. Se recomienda continuar con las clases \~narticulares.

\textbf{Cornejo Tovar Axel Ariel:} Se muestra distraido en clase. Muestra una mediana mejora en su calificaci\'on del examen bimestral respecto al parcial. Es necesario que repase en casa a diario los temas vistos en clase. 

\textbf{Ocampo Camacho Maria Jose:} Se muestra distra\'ida y algo renuente a entender la clase argumentando que no es buena para las matem\'aticas. La calificaci\'on de su examen bimestral es mejor que la del examen parcial. A\'un as\'i es necesario que en casa se le inculque el gusto por las matem\'aticas. Se recomienda repasar los temas vistos en clase diaria\~nente en casa.

\textbf{Virgili Cortes Anna Montserrat:} Tiene dificultad para entender los temas de matem\'aticas. Se recomienda tomar clases particulares dos veces por semana hasta lograr un desempe\~no adecuado. En clase se le estar\'a preguntando constantemente acerca de los temas vistos para lograr un mejor desempe\~no. En su examen bimestral muestra un retroceso respecto al examen parcial.

\vspace{5mm}

{\Large Calificaciones entre 6 Y 7}
\vspace{3mm}

01 Avila Ruiz Evelyn Daniela: Su desempe\~no es medio. Necesita reforzar conocimientos correspondientes a manejo y operaciones con fracciones. 
11 Hernandez Avalos Estefany: Su desempe\~no es medio. La calificaci\'on de su examen bimestral fue superior a la del parcial lo cual indica que va mejorando su desempe\~no. Se le estar\'an preguntando las tablas de multiplicar pues tiene deficiencias (no severas). En clase se le estar\'a preguntando constantemente sobre los procesos que debe seguir para obtener los resultados correctos.
24 Villegas Perez Maximiliano: Tuvo una mejora bastante significativa en su examen bimestral respecto al parcial lo cual indica que est\'a mejorando su desempe\~no. Se recomienda seguir repasando constantemente en casa con lo que podr\'a obtener una buena calificaci\'on en el segundo bimestre.


\section*{\sc Primeoro de secundaria, grupo C}

{\Large Calificaciones menores a 6}
\vspace{3mm}

\textbf{De Armero Hinojosa Renata Sof\'ia:} Presenta dificultad para la comprensi\'on de los temas vistos en clase. Las calificaciones de ambos ex\'amenes son reporbatorias por lo cual se recomienda tomar clases de regularizaci\'on de matem\'aticas dos veces por semana. Es necesario que memorice las tablas de multiplicar. Se le estar\'a pidiendo su participaci\'on constatemente en clase para lograr que mejore su desempe\~no.

\textbf{Dominguez Villase\~nor Juan Pablo:} Las calificaciones de ambos ex\'aemens, parcial y bimestral, son reprobatorias. Se recomienda tomar clases particulares de matem\'aticas para lograr su regularizaci\'on en los temas. En clase se le estar\'a pidiendo su participaci\'on constante para as\'i asegurar una mejora en su desempe\~no. Es necesario que entregue sus ex\'amenes firmados ya que no lo ha hecho.

\textbf{Duran Portillo Frida Fernanda:} Muestra un retroceso en su examen bimestral respecto al parcial. Es necesario que haga a conciencia todos sus trabajos y que en caso de haber errores los corrija para que \'estos ayuden a meorar su desempe\~no y tengan valor en su calificaci\'on. Se le estar\'a pidiendo su participaci\'on en clase de forma constant\'i.

\textbf{Ehnis Borja Rodrigo Eugenio:} Muestra retroceso en su examen bimestral respecto al parcial. Es necesario que entrege todos sus trabajos completos en la fecha indicada. Adem\'as es necesario que en la medida posible no se ausente de forma frecuente para evitar as\'i que pierda las explicaciones que se dan en clase. Es necesario que se aprenda las tablas de multiplicar. Se le estar\'a pidiendo su participaci\'on constantemente e clase para forzar a que mejore su desempe\~no.

\textbf{Hernandez Gonzalez Veronica:} Es que se ponga a trabajar de forma constante en la materia de matem\'aticas. Compa\~neros de ella han se\~nalado qeu frecuentemente les pide las tareas/trabajos para hacer una copia, o bien que hace los trabajos de \'ultima hora y de forma incorrecta. Se recomienda que tome clases de regularizaci\'on de matem\'aticas dos veces por semana. Es urgente\~nque trabaje en casa a diario los temas vistos en clase.

\textbf{Jaime Flores Aracely:} Muestra mejora en su examen bimestral respecto al examen parcial. Es necesario que repase a diario en casa los temas vistos en clase. Es necesario que corrija los ejercicios trabajados en clase que tienen errores para as\'i lograr un mejor desempe\~no y adem\'as que entos cuenten en su calificaci\'on

\textbf{Marquez Amaro Leigh Ann:} Se distrae con mucha facilidad; su atenci\'on va mejorando aunque a\'un hay que practicarla. Ambas calificaciones en sus ex\'amenes no son aprobatorias. Se recomienda repasar a diaro en casa los temas v\~nstos en clase. Adem\'as es necesario que haga sus trabajos a conciencia, que los entrege completos y corregidos a tiempo.


\end{document}




