\newcommand{\seccion}{SECUNDARIA INCORPORADA A LA SEG }
\newcommand{\descripcion}{{\sc Acta de Conducta}}
\newcommand{\grado}{}
\newcommand{\ciclo}{Ciclo escolar: 2015--2016}
\newcommand{\papel}{letterpaper} %letterpaper, legalpaper ...
\newcommand{\fecha}{2 de febrero de 2016}

\author{M. en C. Reinaldo Zapata}

\documentclass[11pt]{article}
\usepackage[\papel]{geometry}

\title{\vspace{-1cm}\flushleft \seccion \\ \descripcion \\  \grado \\ \ciclo}

\newcommand\BackgroundLogo{
\put(160,285){
\parbox[b][\paperheight]{\paperwidth}{%
\vfill
\centering
\includegraphics[width=5cm,height=2.5cm,keepaspectratio]{/Users/reinaldo/Documents/clases/jassa/logo}%
\vfill
}}}

% \hyphenation{con-ti-nua-ci\'on}

\usepackage{enumitem}
\usepackage[T1]{fontenc} %fuentes
\usepackage{lmodern} %fuente mejorada
\usepackage[spanish]{babel}
\decimalpoint
\usepackage{fullpage}
\usepackage{graphicx}
\usepackage{eso-pic}
% \usepackage{multirow}
% \usepackage{subfigure}
% \usepackage{tikz}
% \usepackage{hyperref} 
% \usepackage{color}
% \usepackage{multicol}
% \usepackage{tikz}
% \usetikzlibrary{shapes.geometric}



\usepackage[leqno,fleqn]{amsmath}
\makeatletter
  \def\tagform@#1{\maketag@@@{#1\@@italiccorr}}
\makeatother
\renewcommand{\theequation}{\fbox{\textbf{\arabic{equation}}}}


\begin{document}
\AddToShipoutPicture*{\BackgroundLogo}
\ClearShipoutPicture
\date{\fecha}
\maketitle


Este documento es un acta del comportamiento de la alumna Ana Daniela  Rocha
Rizo debido a las actitudes mostradas de forma reciente en el sal\'on durante la
clase de matem\'aticas del jueves 28 de enero de 2016.

\vspace{5mm}

El pasado d\'ia 28 le llam\'e  en primera instancia la atenci\'on en dos
ocasiones por interrumpir el flujo de la clase al hacer comentarios, en voz
alta, acerca de situaciones que no correspond\'ian al tema en desarrollo.
Posteriormente le llam\'e la atenci\'on debido a que estaba dando fuertes
palmadas en la espalda a una de sus compa\~neras; ante ello argument\'o que su
compa\~nera se estaba ahogando y que ella s\'olo quer\'ia ayudar.
    
\vspace{5mm}

Posterior a esta situaci\'on Daniela decidi\'o entrar al ba\~no sin pedir
autorizaci\'on. Despu\'es de salir se recost\'o boca abajo en uno de los
pasillos que dividen las filas de bancas. Cuando le cuestion\'e sobre su
comportamiento argument\'o que estaba buscando su pluma. Ante esta situaci\'on
le hice saber que la clase ser\'ia interrumpida hasta que ella decidiera dejar
de llamar la atenci\'on y tomara una postura y actitud adecuados para poder
continuar.

\vspace{5mm}

Al momento de sentarse en su lugar adopt\'o una actitud enojada y retadora
debido a lo cual fue necesario mencionarle que su actitud en clase no era la
correcta y que no se trataba de tener una lucha constante viendo quien ced\'ia
primero, si el profesor o los alumnos, haciendo \'enfasis en el hecho de que las
reglas de conducta ya est\'an bien establecidas y que el profesor es quien
dirige la clase.

\vspace{5mm}

Es importante mencionar que en ocasiones anteriores se le hab\'ia llamado la
atenci\'on en repetidas ocasiones pidi\'endole evitar la interrupci\'on del
flujo de trabajo as\'i como respeto hacia sus compa\~neros. Debido a ello se le
dej\'o de pie por periodos de tiempo distintos en dos sesiones.

\vspace{5mm}

En la clase psterior a estos sucesos, correspondiente al martes 2 de febrero,
Daniela present\'o mejora en su comportamiento.

\end{document}










